\documentclass[preprint, 3p,
authoryear]{elsarticle} %review=doublespace preprint=single 5p=2 column
%%% Begin My package additions %%%%%%%%%%%%%%%%%%%

\usepackage[hyphens]{url}

  \journal{Geographical Analysis} % Sets Journal name

\usepackage{lineno} % add
  \linenumbers % turns line numbering on

\usepackage{graphicx}
%%%%%%%%%%%%%%%% end my additions to header

\usepackage[T1]{fontenc}
\usepackage{lmodern}
\usepackage{amssymb,amsmath}
\usepackage{ifxetex,ifluatex}
\usepackage{fixltx2e} % provides \textsubscript
% use upquote if available, for straight quotes in verbatim environments
\IfFileExists{upquote.sty}{\usepackage{upquote}}{}
\ifnum 0\ifxetex 1\fi\ifluatex 1\fi=0 % if pdftex
  \usepackage[utf8]{inputenc}
\else % if luatex or xelatex
  \usepackage{fontspec}
  \ifxetex
    \usepackage{xltxtra,xunicode}
  \fi
  \defaultfontfeatures{Mapping=tex-text,Scale=MatchLowercase}
  \newcommand{\euro}{€}
\fi
% use microtype if available
\IfFileExists{microtype.sty}{\usepackage{microtype}}{}
\usepackage[]{natbib}
\bibliographystyle{plainnat}

\ifxetex
  \usepackage[setpagesize=false, % page size defined by xetex
              unicode=false, % unicode breaks when used with xetex
              xetex]{hyperref}
\else
  \usepackage[unicode=true]{hyperref}
\fi
\hypersetup{breaklinks=true,
            bookmarks=true,
            pdfauthor={},
            pdftitle={Reproducibility of research during COVID-19: examining the case of population density and the basic reproductive rate from the perspective of spatial analysis},
            colorlinks=false,
            urlcolor=blue,
            linkcolor=magenta,
            pdfborder={0 0 0}}

\setcounter{secnumdepth}{0}
% Pandoc toggle for numbering sections (defaults to be off)
\setcounter{secnumdepth}{0}


% tightlist command for lists without linebreak
\providecommand{\tightlist}{%
  \setlength{\itemsep}{0pt}\setlength{\parskip}{0pt}}






\begin{document}


\begin{frontmatter}

  \title{Reproducibility of research during COVID-19: examining the case
of population density and the basic reproductive rate from the
perspective of spatial analysis}
    \author[McMaster University]{Antonio Paez%
  \corref{cor1}%
  }
   \ead{paezha@mcmaster.ca} 
      \affiliation[McMaster University]{School of Earth, Environment and
Society, 1280 Main St West, Hamilton, Ontario L8S 4K1 Canada}
    \cortext[cor1]{Corresponding author}
  
  \begin{abstract}
  The emergence of the novel SARS-CoV-2 coronavirus and the global
  COVID-19 pandemic in 2019 led to explosive growth in scientific
  research. Alas, much of the research in the literature lacks
  conditions to be reproducible, and recent publications on the
  association between population density and the basic reproductive
  number of SARS-CoV-2 are no exception. Relatively few papers share
  code and data sufficiently, which hinders not only verification but
  additional experimentation. In this paper, an example of reproducible
  research shows the potential of spatial analysis for epidemiology
  research during COVID-19. Transparency and openness means that
  independent researchers can, with only modest efforts, verify findings
  and use different approaches as appropriate. Given the high stakes of
  the situation, it is essential that scientific findings, on which good
  policy depends, are as robust as possible; as the empirical example
  shows, reproducibility is one of the keys to ensure this.

  This paper is now published in Geographical Analysis
  (\url{https://doi.org/10.1111/gean.12307})
  \end{abstract}
    \begin{keyword}
    keyword1 \sep 
    keyword2
  \end{keyword}
  
 \end{frontmatter}

Please make sure that your manuscript follows the guidelines in the
Guide for Authors of the relevant journal. It is not necessary to
typeset your manuscript in exactly the same way as an article, unless
you are submitting to a camera-ready copy (CRC) journal.

For detailed instructions regarding the elsevier article class, see
\url{https://www.elsevier.com/authors/policies-and-guidelines/latex-instructions}

\renewcommand\refname{References}
\bibliography{bibliography.bib}


\end{document}
