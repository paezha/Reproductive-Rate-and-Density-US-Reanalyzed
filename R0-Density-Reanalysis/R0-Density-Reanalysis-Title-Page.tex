\documentclass[]{elsarticle} %review=doublespace preprint=single 5p=2 column
%%% Begin My package additions %%%%%%%%%%%%%%%%%%%
\usepackage[hyphens]{url}

  \journal{Landscape and Urban Planning} % Sets Journal name


\usepackage{lineno} % add
  \linenumbers % turns line numbering on
\providecommand{\tightlist}{%
  \setlength{\itemsep}{0pt}\setlength{\parskip}{0pt}}

\usepackage{graphicx}
\usepackage{booktabs} % book-quality tables
%%%%%%%%%%%%%%%% end my additions to header

\usepackage[T1]{fontenc}
\usepackage{lmodern}
\usepackage{amssymb,amsmath}
\usepackage{ifxetex,ifluatex}
\usepackage{fixltx2e} % provides \textsubscript
% use upquote if available, for straight quotes in verbatim environments
\IfFileExists{upquote.sty}{\usepackage{upquote}}{}
\ifnum 0\ifxetex 1\fi\ifluatex 1\fi=0 % if pdftex
  \usepackage[utf8]{inputenc}
\else % if luatex or xelatex
  \usepackage{fontspec}
  \ifxetex
    \usepackage{xltxtra,xunicode}
  \fi
  \defaultfontfeatures{Mapping=tex-text,Scale=MatchLowercase}
  \newcommand{\euro}{€}
\fi
% use microtype if available
\IfFileExists{microtype.sty}{\usepackage{microtype}}{}
\bibliographystyle{elsarticle-harv}
\ifxetex
  \usepackage[setpagesize=false, % page size defined by xetex
              unicode=false, % unicode breaks when used with xetex
              xetex]{hyperref}
\else
  \usepackage[unicode=true]{hyperref}
\fi
\hypersetup{breaklinks=true,
            bookmarks=true,
            pdfauthor={},
            pdftitle={Population density and the spread of the COVID-19 pandemic: a reproducible research example},
            colorlinks=false,
            urlcolor=blue,
            linkcolor=magenta,
            pdfborder={0 0 0}}
\urlstyle{same}  % don't use monospace font for urls

\setcounter{secnumdepth}{0}
% Pandoc toggle for numbering sections (defaults to be off)
\setcounter{secnumdepth}{0}

% Pandoc citation processing

% Pandoc header
\usepackage{setspace}\doublespacing



\begin{document}
\begin{frontmatter}

  \title{Population density and the spread of the COVID-19 pandemic: a
reproducible research example}
    \author[McMaster University]{Antonio Paez\corref{1}}
   \ead{paezha@mcmaster.ca} 
      \address[McMaster University]{School of Earth, Environment and
Society, 1280 Main St West, Hamilton, Ontario L8S 4K1 Canada}
      \cortext[1]{Corresponding Author}
  
  \begin{abstract}
  The emergence of the novel SARS-CoV-2 coronavirus and the global
  COVID-19 pandemic has led to explosive growth in scientific research.
  Of interest has been the associations between population density and
  the spread of the pandemic. In this paper, population density and the
  basic reproductive number of SARS-CoV-2 are examined in an example of
  reproducible research. Given the high stakes of the situation, it is
  essential that scientific activities, on which good policy depends,
  are as transparent and reproducible as possible. Reproducibility is
  key for the efficient operation of the self-correction mechanisms of
  science. Transparency and openness means that the same problem can,
  with relatively modest efforts, be examined by independent researchers
  who can verify findings, and bring to bear different perspectives,
  approaches, and methods---sometimes with consequential changes in the
  conclusions, as the empirical example of the spread of COVID-19 in the
  US shows.
  \end{abstract}
  
 \end{frontmatter}




\end{document}

